% My first document
\documentclass[12pt]{scrartcl}
\title{First steps in \LaTeX}
\author{John Doe
	\thanks{Third year CMP student}}
% End of preamble, beginning of document
	
% end of pramble, beginning of document
\begin{document}
	\maketitle
	\begin{abstract}
	The Life of Brian is not really interesting but I am
	paid to do this job. No one would remember Brian’s
	life if not for a mad lecturer who used it as an
	illustrative example in a tutorial at the University
	of East Anglia.
	\\
	\section{Introduction}
	Everyone knows, even the dumbest reader, that Brian was
	an ordinary bloke.
	\section{Infancy}
	As Albert Einstein, Brian was a late developer, except
	that, contrary to Einstein, Brian never developed very
	much.
	
	\subsection{Before the age of six}
	Absolutely nothing happened to Brian until the age of
	six\ldots{} and not much after.
	\subsubsection{Do we need to elaborate?}
	No!
	\section*{Acknowledgement}
	The help of the Brian Trust was unvaluable in writing
	this paper.
	
	As Albert Einstein, Brian was a late developer, except
	that, contrary to Einstein, Brian \emph{never} developed very
	much\footnote{According to his tutor Albert Einstein.}.
	\end{abstract}
	
	When you write a
	paragraph with \LaTeX{} you do not need to
take care of it.
% new paragraph when one or more blank lines

When \LaTeX{} encounters a blank line
the current paragraph finishes.
Next time it finds some text it starts a new paragraph.

List of things to do:\\
Learn \LaTeX, practice with \LaTeX ,
become a \LaTeX{}pert,
write a book on \LaTeX ,
bore the pants of students with \LaTeX.

I am 100\% sure you’ll have no problem
using \LaTeX{} \ldots{} I am prepared to bet
1\pounds on it (I am not mad!).

"This is an incorrectly formatted quote".
‘‘This is a correctly formatted quote’’

A simple table:
\begin{center}
	\begin{tabular}{clr}
		centred & left & right \\
		C & L & R \\
		111 & 222 & 333
	\end{tabular}
\end{center}


A simple table:
\begin{center}
	\begin{tabular}{clr} \hline
		centred & left & right \\ \hline\hline
		C & L & R \\
		111 & 222 & 333 \\ \hline
	\end{tabular}
\end{center}

\begin{center}
	\begin{tabular}{|l|r|r|r|r|} \cline{2-5}
		\multicolumn{1}{c}{} & \multicolumn{2}{|c|}{CPU} &
		\multicolumn{2}{|c|}{Iteration} \\ \hline
		Test problem & average & maximum & average & maximum \\
		\hline \hline
		Norwich100 & 10.1 & 121.2 & 212 & 2333 \\
		London123 & 7.2 & 61.3 & 110 & 650 \\
		\hline
	\end{tabular}
\end{center}
\end{document}